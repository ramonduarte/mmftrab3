%%%%%%%%%%%%%%%%%%%%%%%%%%%%%%%%%%%%%%%%%
% Lachaise Assignment
% LaTeX Template
% Version 1.0 (26/6/2018)
%
% This template originates from:
% http://www.LaTeXTemplates.com
%
% Authors:
% Marion Lachaise & François Févotte
% Vel (vel@LaTeXTemplates.com)
%
% License:
% CC BY-NC-SA 3.0 (http://creativecommons.org/licenses/by-nc-sa/3.0/)
% 
%%%%%%%%%%%%%%%%%%%%%%%%%%%%%%%%%%%%%%%%%

%----------------------------------------------------------------------------------------
%	PACKAGES AND OTHER DOCUMENT CONFIGURATIONS
%----------------------------------------------------------------------------------------

\documentclass{article}

\input{structure.tex} % Include the file specifying the document structure and custom commands

%----------------------------------------------------------------------------------------
%	ASSIGNMENT INFORMATION
%----------------------------------------------------------------------------------------

\title{MAE001: Modelagem Matemática em Finanças I} % Title of the assignment
\date{Universidade Federal do Rio de Janeiro (UFRJ) --- \today} % University, school and/or department name(s) and a date

\author{Ramon Duarte de Melo\\ \texttt{ramonduarte@poli.ufrj.br} % Author name and email address
\\ \\ Alex Teixeira da Silva\\ \texttt{alextx@poli.ufrj.br}} % Author name and email address


%----------------------------------------------------------------------------------------

\begin{document}

\maketitle % Print the title

%----------------------------------------------------------------------------------------
%	INTRODUCTION
%----------------------------------------------------------------------------------------

\section*{Introdução} % Unnumbered section

O objetivo do Projeto III é implementar, avaliar e comparar o modelo de Black-Scholes com os dados fornecidos pelo mundo real, realizando comparações de cunho matemático-estatístico e produzindo gráficos com tais observações acerca da volatilidade implícita, do preço de corte (\emph{strike price} e dos recortes temporais). 

Para tal, foi utilizada a linguagem \emph{Python 3.6.7}, com os módulos \emph{numpy} (métodos numéricos), \emph{pandas} (manipulação de dados), \emph{scipy} (fórmulas científicas) e \emph{matplotlib.pyplot} (visualização de dados).

Os dados utilizados para a confecção das comparações foi raspado da web utilizando as ferramentas \emph{bs4} (web parsing), \emph{lxml} (HTML parsing) e \emph{re} (expressões regulares). O programa requer a instalação destes módulos, mas possui uma ferramenta de instalação automatizada das dependências (\emph{pipenv}). 

As fontes dos dados são a B3 (Brasil Bolsa Balcão, operadora da Bolsa de Valores de São Paulo) e o jornal Valor Econômico. Todos os dados são referentes ao mercado logo após o fechamento (17:00) do dia 6º de junho de 2019. O ativo utilizado, \textbf{PETR4 ON}, fechou o pregão a $R\$ $ $29,85$.


O código utilizado neste trabalho, bem como o deste relatório e as imagens geradas, foi aberto e disponibilizado publicamente no repositório https://github.com/ramonduarte/mmftrab3.


%----------------------------------------------------------------------------------------
%	PROBLEM 1
%----------------------------------------------------------------------------------------

\section*{Atividade \emph{a}} % Numbered section

Nesta atividade, foi implementado o modelo de Black-Scholes da seguinte forma:

\begin{enumerate}
	\item Conforme definição, o modelo de Black-Scholes define preços de opções como uma função do preço do ativo subjacente $S$, do preço de execução $K$ (também chamado de \emph{strike price}), da taxa de renda fixa $r$ (aqui foi usada a Taxa Selic), do tempo até o prazo de execução $t$ e da volatilidade $\sigma$.
	
	\[V = BS(S,K,r,T,\sigma)\]

	\item Como todos os valores acima são conhecidos exceto a volatilidade, é possível descrever o preço da opção como uma função somente dela.
		
	\[V = BS(\sigma)\]

	\item Para tanto, é necessário descobrir o valor da opção $V$. Como o modelo de Black-Scholes apoia-se na premissa da \emph{informação perfeita}, podemos considerar que toda opção negociada na bolsa de valores está decentemente precificada.
	\item No entanto, não é possível manipular a equação de Black-Scholes para isolar o valor da volatilidade, de forma que a única maneira de calculá-la é utilizar a técnica de bisseção. 
\end{enumerate}

Com os valores obtidos, foram gerados gráficos $K \times T$ para opções de compra e venda do ativo \textbf{PETR4\ ON} para exercício em três datas: 17/06/2019, 19/08/2019 e 20/01/2020, respectivamente.
As opções não foram separadas entre derivativos europeus e americanos.
No entanto, somente foram incluídos derivativos europeus e americanos.
Contratos exóticos e/ou dependentes de caminho não foram incluídos.



\begin{figure}[]
	\includegraphics[width=0.9\linewidth]{Figure_0.png}
	\centering
	
	\caption{Gráfico $K \times \sigma$ para opções do ativo \textbf{PETR4 ON} com data de exercício 17/06/2019.}
	\label{}
\end{figure}

\begin{figure}[]
	\includegraphics[width=0.9\linewidth]{Figure_1.png}
	\centering
	
	\caption{Gráfico $K \times \sigma$ para opções do ativo \textbf{PETR4 ON} com data de exercício 19/08/2019.}
	\label{}
\end{figure}

\begin{figure}[]
	\includegraphics[width=0.9\linewidth]{Figure_2.png}
	\centering
	
	\caption{Gráfico $K \times \sigma$ para opções do ativo \textbf{PETR4 ON} com data de exercício 20/01/2020.}
	\label{}
\end{figure}

%----------------------------------------------------------------------------------------
%	PROBLEM 2
%----------------------------------------------------------------------------------------


\section*{Atividade \emph{b}}

Para esta atividade, o processo utilizado foi o mesmo do da anterior.
As únicas mudanças foram:

\begin{itemize}
	\item Uma subbiblioteca do \emph{matplotlib}, externa ao \emph{pyplot}, teve de ser utilizada para gerar o gráfico em 3D, porque o \emph{pyplot} não gera gráficos 3D interativos, necessários para a escolha da melhor perspectiva.
	\item Todas as opções encontradas para o ativo \textbf{PETR4 ON} foram utilizadas, desde que estivessem precificadas após o fechamento do pregão da Bolsa de Valores de São Paulo em 06/06/2019.
\end{itemize}

\begin{figure}[H]
	\includegraphics[width=\linewidth]{Figure_3.png}
	\centering
	
	\caption{\textbf{Superfície de volatilidade}: séries $K \times \sigma$ separadas por data de exercício.}
	\label{}
\end{figure}

%----------------------------------------------------------------------------------------
%	PROBLEM 2
%----------------------------------------------------------------------------------------

\section*{Atividade c}

Com exceção da \emph{Figura 3}, foi possível enxergar o \emph{smile} em todos os gráficos plotados.
Como o ativo \textbf{PETR4 ON} é bastante negociado na bolsa de valores, suas distribuições tendem a se aproximarem do previsto pelo modelo teórico.
É justamente por isso que, na \emph{Figura 3}, a observação do \emph{smile} é mais difícil, pois as opções com exercício nesta data - mais distante - são menos negociadas e, portanto, não formam pontos suficientes para a construção da curva com visibilidade.

Na \emph{Figura 4}, inclusive, é possível notar a formação de várias curvas deste tipo ao longo das datas de exercício, bem como seu rareamento conforme o aumento do prazo.


\end{document}
